\documentstyle [12pt]{article}

\textheight22truecm
\textwidth16.5truecm
\setlength{\oddsidemargin}{0cm}
\setlength{\evensidemargin}{0cm}
\setlength{\topmargin}{0cm}


\def\epi{{\rm epi \,}}
\def\nr{\par \noindent}
\def\diag{{\rm diag \,}}
\def\inter{{\rm int \,}}
\def\trace{{\rm Trace \,}}
\def\lin{{\rm Lin \,}}
\def\conv{{\rm Conv \,}}
\def\beq{\begin{equation}}
\def\eeq{\end{equation}}
\newcommand{\rint}{{\rm rint\,}}
\newcommand{\N}{\parallel}
\newcommand{\A}{\mid}

\newtheorem{theorem}{Theorem}[section]
\newtheorem{lemma}{Lemma}[section]
\newtheorem{corollary}{Corollary}[section]
\newtheorem{condition}{Condition}[section]
\newtheorem{conjecture}{Conjecture}[section]
\newtheorem{proposition}{Proposition}[section]
\newtheorem{assumption}{Assumption}[section]
\newtheorem{definition}{Definition}[section]
\newtheorem{property}{Property}[section]
\newtheorem{remark}{Remark}[section]
\newcommand{\proof}{\bf Proof: \rm \nr}
\newcommand{\qed}{\hfill $\Box$ \nr \medskip}
\newcommand{\half}{\mbox{${1 \over 2}$}}
\renewcommand\arraystretch{1}
\def\ba{\begin{array}}
\def\ea{\end{array}}
\def\beann{\begin{eqnarray*}}
\def\eeann{\end{eqnarray*}}
\def\bea{\begin{eqnarray}}
\def\eea{\end{eqnarray}}

\def\BT{\begin{theorem}}
\def\ET{\end{theorem}}
\def\BL{\begin{lemma}}
\def\EL{\end{lemma}}
\def\BC{\begin{corollary}}
\def\EC{\end{corollary}}
\def\BA{\begin{assumption}}
\def\EA{\end{assumption}}


\def\la{\langle}
\def\ra{\rangle}

\def\theequation{\thesection.\arabic{equation}}

\begin{document}
\title{Methods for Nonsmooth Convex Minimization}
\author{INMA 2460: \\
Nonlinear Programming,\\
Exercise $\#$ 2}

%\date{March, 1996}

%\date{March, 2004}

%\date{April, 2015}

%\date{April, 2016}
%\date{April 2017}
%\date{April 2018}
%\date{April 2019}
\date{April 2020}

\maketitle

\section{Motivation}

The goal of this exercise consists in practical
implementation and comparison of two nonsmooth convex
optimization methods. The students are asked to create the
corresponding computer programs, run several series of
tests and write down a report with the analysis of the
results. The formal requirements are as follows.
\begin{itemize}
\item
{\bf Schedule:} The final report has to be presented no
later than last Friday in May 2020.
\item
{\bf Marks:} This exercise is not obligatory.
However, we strongly recommend to do it
in the best possible way since
it adds up to 5 points at
the final examination.
\item
{\bf Software:} The students can use any
programming language ({\em C, Fortran}, etc.) or
environment ({\em MatLab}).
\item
{\bf Final Report.} This report is usually composed by the
following parts:
\begin{enumerate}
\item
Description of the problems and the numerical methods.
\item\label{strat}
Description of the set of test problems and
the testing strategy.
\item\label{an}
Analysis of the results.
\item
Appendix (listing of the code and full computation
results)
\end{enumerate}
Items \ref{strat} and \ref{an} are of the most importance
for the final evaluation.
\end{itemize}


\section{Problem Formulation}

The problem formulation is as follows:
\beq\label{prob}
\min\limits_{x \in R^n} \; f(x),
\eeq
where
$f$ is a nondifferentiable convex function.
\BA
For simplicity, we assume that the minimum $x^*$ of
problem (\ref{prob}) exists, and we know an estimate
$\rho$:
$$
\N x_0 - x^* \N \leq \rho,
$$
where $x_0$ is the starting point of the method.
\EA

\section{Methods (Lecture 8)}

In this exercise, it is necessary to test two numerical
methods.

\subsection{Subgradient Method}
$$
\ba{rcl}
x_0 & \in & R^n;\\
\\
x_{k+1} & = & x_k - {\rho \over \sqrt{k+1}}{g_k \over \|
g_k \| },
\quad k \geq 0,
\ea
$$
where $g_k \in \partial f(x_k)$.

\subsection{Ellipsoid Method}
$$
\ba{rclrcl}
x_0 & \in & R^n, & x_{k+1} & = & x_k - {1 \over n + 1}
\cdot { H_k g_k \over \la H_k g_k, g_k \ra^{1/2}},\\
\\
H_0 & = & \rho^2 I_n, & H_{k+1} & = & {n^2 \over n^2 - 1}
\left(H_k - { 2 \over n + 1} \cdot { H_k g_k g_k^T H_k
\over \la H_k g_k, g_k \ra }\right), \quad k \geq 0,
\ea
$$
where $I_n$ is the unit $n \times n$ matrix and
$g_k \in \partial f(x_k)$.

\section{Computation of the subgradients (Lecture 7)}\label{sc-set}

Recall that vector $g \in R^n$ is called {\em subgradient}
of function $f(\cdot)$ at point $x_0$, if for any $x \in
R^n$ we have
$$
f(x) \geq f(x_0) + \la g, x - x_0 \ra.
$$
For nondifferentiable convex functions, the subgradient is
not always unique. The set of all subgradients at $x_0$ is
denoted by $\partial f(x_0)$.

The following rules can be applied for computing the
subgradients.

1. If $f(\cdot)$ is differentiable at $x_0$, then the
subdifferential consists of a single vector, the gradient:
$$
\partial f(x_0) \equiv \{ f'(x_0) \}.
$$

2. If $f(x) = \alpha f_1(x) + \beta f_2(x)$ with $\alpha$,
$\beta \geq 0$, then
$$
\partial f(x_0) = \alpha \partial f_1(x_0) + \beta \partial f_2(x_0).
$$
This means that for any $g_1 \in \partial f_1(x_0)$ and
$g_2 \in \partial f_2(x_0)$, we have
$$
\alpha g_1 + \beta g_2 \in \partial f(x_0).
$$

3. If $f(x) = \max \{ f_1(x), f_2(x) \}$, then
$$
\partial f(x_0) = \left\{
\ba{cl}
\partial f_1(x_0), & \mbox{if } f_1(x_0) > f_2(x_0),\\
\\
\partial f_2(x_0), & \mbox{if } f_1(x_0) < f_2(x_0),\\
\\
\mbox{Conv}\{ \partial f_1(x_0), \partial f_2(x_0) \}, &
\mbox{if } f_1(x_0) = f_2(x_0).
\ea
\right.
$$
In the latter case, for any $g_1 \in \partial f_1(x_0)$,
$g_2 \in \partial f_2(x_0)$ and $\alpha \in [0,1]$, we
have
$$
\alpha g_1 + (1 - \alpha) g_2 \in \partial f(x_0).
$$


\section{Test Problems}

Any test problems is defined by the choice of
the following objects:
\begin{itemize}
\item
objective function $f(x)$,
\item
starting point $x_0$ for the minimization process,
\item
accuracy of the approximate solution $\epsilon > 0$.
\end{itemize}

In order to have a complete information about the behavior
of numerical method, it is reasonable to generate the test
problems with known optimal solutions. Therefore, we
suggest to use the following strategy.
\begin{enumerate}
\item
Choose the dimension $n \geq 2$ of the space of variables.
\item
Fix the optimal solution of the problem as $x^* = 0 \in
R^n$.
\item
Choose the objective function. We suggest to use the
following family of objective functions:
$$
f(x) = \alpha f_1(x)
+ \beta f_2(x),
$$
where the parameters $\alpha$ and $\beta$ are nonnegative
and
$$
\ba{l}
f_1(x) = \sum\limits_{i=1}^{n-1} \A x^{(i)} \A,\\
\\
f_2(x) = \max\limits_{1 \leq i \leq n} \A x^{(i)} \A - x^{(1)}.
\ea
$$
Then the Lipschitz constant $L$ for the objective function
can be estimated as follows:
$$
L = \alpha \sqrt{n} + 2 \beta.
$$
\item
Choose the starting point $x_0 \in Q$. The important
characteristic of the problem is $\rho = \N x_0 - x^* \N$.
\item
Choose the desired accuracy $\epsilon > 0$. If you
implement the above mentioned strategy for generating the
objective function, then the optimal value of the problem
is always zero. Therefore it is reasonable to introduce in
the minimization scheme a termination criterion $f(x_k)
\leq \epsilon$. Then the number of iterations, which is
necessary to achieve the desired accuracy, can be easily
fixed out.
\end{enumerate}

\section{Testing Strategy}

In the final report, it is necessary to justify a
conclusion on the performance and the {\em sensitivity} of
the methods to the following characteristics:
\begin{itemize}
\item
Desired accuracy $\epsilon$.
\item
Lipschitz constant $L$.
\item
Initial distance to the minimum $\rho$.
\item
Dimension of the space $n$.
\end{itemize}
The typical values of these parameters are presented in
the following table.
$$
\ba{|l|c|r|r|r|}
\hline
                     & \epsilon & L & \rho & n\\ \hline
\mbox{Low/Small}   &  10^{-2} & 10 & 10 & 10\\ \hline
\mbox{Moderate}      &  10^{-4} & 100 & 100 & 100\\ \hline
\mbox{High/Large}  &  10^{-6} & 1000 & 1000 & 1000 \\
\hline
\ea
$$

\end{document}
